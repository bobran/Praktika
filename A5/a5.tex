\documentclass[a4paper,12pt]{article}
\usepackage{czech}
\usepackage[utf8]{inputenc}
\usepackage{a4wide}
\usepackage[dvipdfm]{graphicx}
\usepackage{graphics}
\usepackage{indentfirst}
\usepackage{fancyhdr}
\usepackage{setspace}
\usepackage{amsmath}
\usepackage{amssymb}
\usepackage{epsfig}

%%\usepackage{nopageno}
%%\usepackage{txfonts}
\usepackage[usenames]{color}

\begin{document}
\section{Úkol}
\begin{enumerate}
    \item Proveďte energetickou kalibraci \alpha-spektrometru a určete jeho rozlišení.
    \item Určete absolutní aktivitu kalibračního radioizotopu $^{241}$Am.
    \item Změřte závislost ionizační ztráty \alpha-částic ve vzduchu při normálním 
    tlaku -d$T$/dx= $f(t)$. Srovnejte tuto závislost se závislotí získanou pomocí empirické formule 
    pro dolet \alpha-častic ve vzduchu za normálních podmínek.
    \item Určete energie \alpha-částic vzletujících ze vzorku obsahující izotop $^{239}$Pu a příměs 
    izotopu $^{238}$Pu a porovnejte je s tabelovými hodnotami. Stanovne relativní zastoupení izotopu 
    $^{238}$Pu ve vzorku s přesností lepší než 10 \%, jsou-li T$_{1/2}$($^{238}$Pu)=81.71 yr a 
    T$_{1/2}$($^{239}$Pu)=$24.13\cdot10^3$ yr.
\end{enumerate}

\section{Teoretický úvod}

\section{Měření}
\subsection{Kalibrace}
Za pomoci známé energie \alpha-částic radioizotopu Am jsme provedli kalibraci spektrometru. Naměřené hosnoty jsou v prvním řádku tabulky \ref{mer}. 
Z hodnoty pološířky a vztahu \ref{gamma} jsem stanovil rozlišení spektrometru na
\begin{eqnarray}
\Gamma = 66.48 \mbox{keV}.
\end{eqnarray}

\subsection{Aktivaita vzorku}
Plocha histogramu odpovídá počtu interakcí na detektoru za dobu měření. Naše měření trvalo 500 sekund. Detektor měl tvar kruhu o průměru 
\begin{eqnarray}
d=1.122 \mbox{cm}
\end{eqnarray}
a byl ve vzdálenosti
\begin{eqnarray}
x=3 \mbox{cm}
\end{eqnarray}
od vzorku. Vzorek považujeme za bodový a detektor můžeme aproximovat kulovou úsečí. Pro absolutní aktivitu pak získáme vztah
\begin{eqnarray}
A=\frac{N}{t}=4\pi\frac{N_m}{\Omega_dt}=16x\frac{N_m}{d^2t},
\end{eqnarray}
což dá po dosazení naměřených hodnot
\begin{eqnarray}
A=367045 \mbox{s}^{-1}
\end{eqnarray}

\subsection{Závislost ionizačních ztrát}
Pro provedení kalibrace jsme provedli měření pro různé hodnoty tlaku v komoře. Naměřené hodnoty 
jsou v tabulce \ref{mer}

\begin{table}
$$
\begin{array}{|c|c|c|c|c|c|c|}
\hline
p/\mbox{ATM}&E\mbox{keV}&\mbox{FHHM}&S&\mbox{backg}&\mbox{NET C/S}& \mbox{ERR}\\ \hline
0.0&5485.73&28.23&48132&162&96.264&0.46\\ \hline
0.1&5241.00&39.71&47925&240&95.850&0.46\\ \hline
0.2&4990.52&52.86&48282&127&96.564&0.46\\ \hline
0.3&4733.02&69.41&48231&174&96.462&0.46\\ \hline
0.4&4470.75&84.84&48002&97&96.001&0.46\\ \hline
0.5&4198.63&101.11&48135&139&96.270&0.46\\ \hline
0.6&3917.96&117.67&48437&99&96.874&0.46\\ \hline
0.7&3633.86&136.26&48435&135&96.870&0.46\\ \hline
0.8&3309.94&156.69&48424&162&96.848&0.46\\ \hline
0.9&2953.28&185.95&48197&131&96.394&0.46\\ \hline
1.0&2644.14&211.98&48122&217&96.244&0.46\\ \hline
\end{array}
$$
\caption{Měření energi $\alpha$-částic pro různé hodnoty tlaku}
\label{mer}
\end{table}
Z těchto hodnot následně dopočítáme ionizační ztráty. Hodnoty jsou v tabulce \ref{iz}.

\begin{table}
$$
\begin{array}{|c|c|}
\hline
p/\mbox{ATM}& \Delta T \\ \hline
0.0&0\\ \hline
0.1&244.7\\ \hline
0.2&495.2\\ \hline
0.3&752.7\\ \hline
0.4&1015.0\\ \hline
0.5&1287.1\\ \hline
0.6&1567.8\\ \hline
0.7&1851.9\\ \hline
0.8&2175.8\\ \hline
0.9&2532.4\\ \hline
1.0&2841.6\\ \hline
\end{array}
$$
\caption{Ionizační ztráty $\alpha$-částic v závislosti na tlaku}
\label{iz}
\end{table}

\subsection{Specifické ztráty}
Naměřeným energiím jsem nafitoval polynom s předpisem
\begin{eqnarray}
T(x)=-36.4*x^3+77*x^2-890*x+5494
\end{eqnarray}
Tomuto předpisu odpovídá dle \ref{brag} Braggova křivka
\begin{eqnarray}
h(x)=109.2*x^2-154*x+890
\end{eqnarray}

\subsection{Pu}
Provedl jsem měření pro vzorek ze směsi plutonia izotopů 238 a 239. Pro Jenotlivé píky jsem získal hodnoty uvedené  v tabulce \ref{pu}.

\begin{table}
$$
\begin{array}{|c|c|c|c|c|c|c|}
\hline
&E\mbox{keV}&\mbox{FHHM}&S&\mbox{backg}&\mbox{NET C/S}& \mbox{ERR}\\ \hline
^{239}\mbox{Pu}&5142.42&34.26&106570&321&213.140&0.31\\ \hline
^{238}\mbox{Pu}&5484.98&40.25&917&5&1.834&3.27\\ \hline
\end{array}
$$
\caption{Hodnoty naměřené pro smět Pu izotopů 238 a239}
\label{pu}
\end{table}

Poměr izotopů 238 ku 239 následně získáme po dosazení do vtahu \ref{}
\begin{eqnarray}
P=29.14
\end{eqnarray}

\section{Diskuze}

\section{Závěr}


\end{document}
