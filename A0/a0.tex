hlavička

\section{Úkol}
\begin{enumerate}
    \item Cílem studované úlohy je seznámit posluchače s vlastnostmi spekter gama záření získaných polovodičovým spektrometrem.
    Měření se provádí na spektrometru KJF s GeLi detektorem o objemu aktivní oblasti 55 cm$^3$ (průměr čela detektoru je 70 mm). 
    Měření je prováděno se zářiči s jednoduchým spektrem gama záření: $^137$Cs (E = 661.55 keV), $^60$Co (E = 1173.27 keV) a 
    $^24$Na (E = 1368.63 a 2754.03 keV), které jsou současně používány ke kalibraci spektrometru. K nastavení geometrue zážič-detektor 
    se používá jednoduchý nosič zářičů umožňující volbu různé geometrie.

\section{Teoretický úvod}

\section{Měření}
\subsection{Kalibrace}
Nejprve jsem provedli kalibraci spektrometru za pomoci Ra, jehož spektrum je dobře známo.
Ke kalibraci jsme využili 6 bodů spektra.

\subsection{Cs}
Dále jsme na spektrometr umístili Cs izotop 137 a pozorovali vzniklé spektrum, které je vidět na obrázku \ref{spect1}. 
Na tomto spektru je vidět jediný dominantní pík, který odpovídá energii gama zážení Cs. Tato energie je 
$$
E=  \mbox{keV}.
$$

Dále můžeme na spektru pozorovat projevy Comptnova rozptylu. Comptnovy hrany se nachází na energiích
$$
E_{h1}= \mbox{keV}
E_{h2}= \mbox{keV}
$$
Hrana zpětného odrazu je na energii
$$
E_{hz}= \mbox{keV}
$$
Dle teorie by hodnoty těchto hran měli být
$$
E_{th1}=477,33 \mbox{keV}
E_{th2}=554.58 \mbox{keV}
$$

\subsection{sul}
Dále jsme pozorovali sůl ozařovanou neutronovým zářením. Tak v ní vznikly izotopy $^{24}$Na a radioaktivní izotop Cl.
Tam se mimo Comptnova rozpztylu navíc projevil efekt, při kterém z dvojice fotnů vzniká pár elektron, pozitron. Píky tohoto 
efektu jsou vyznačeny spolu s dalšímy body na obrázku \ref{spect2}. Jejich energetické hodnoty jsou
$$
E_{Na}=\mbox{keV}
E_{NaSEP}=\mbox{keV}
E_{NaDEP}=\mbox{keV}
E_{Na2}=\mbox{keV}
E_{Na2SEP}=\mbox{keV}
E_{ani}=\mbox{keV}
$$
Pro srovnání jsou teoretické energie 
$$
E_{tNa}=2753.03  \mbox{keV}
E_{tNaSEP}=2242.03  \mbox{keV}
E_{tNaDEP}=1731.03  \mbox{keV}
E_{tNa2}=1173.23  \mbox{keV}
E_{tNa2SEP}=622.23  \mbox{keV}
E_{tani}=1022  \mbox{keV}
$$
Pro úplnost jsem určil i hodnoty comptnových hran
$$
E_{hNa}=\mbox{keV}
E_{h2Na}=\mbox{keV}
E_{hNa2}=\mbox{keV}
E_{h2Na2}=\mbox{keV}
E_{hzNa}=\mbox{keV}
E_{hzNa2}=\mbox{keV}
$$
Pro srovnání opět uvádím teoretické hodnoty
$$
E_{thNa}=2520.22 \mbox{keV}
E_{th2Na}=2631.94 \mbox{keV}
E_{thNa2}=963.46\mbox{keV}
E_{th2Na2}=1058.06\mbox{keV}
$$


