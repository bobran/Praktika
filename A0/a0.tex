\documentclass[a4paper,12pt]{article}
\usepackage{czech}
\usepackage[utf8]{inputenc}
\usepackage{a4wide}
\usepackage[dvipdfm]{graphicx}
\usepackage{graphics}
\usepackage{indentfirst}
\usepackage{fancyhdr}
\usepackage{setspace}
\usepackage{amsmath}
\usepackage{amssymb}
\usepackage{epsfig}

%%\usepackage{nopageno}
%%\usepackage{txfonts}
\usepackage[usenames]{color}

\begin{document}
\section{Úkol}
\begin{enumerate}
    \item Cílem studované úlohy je seznámit posluchače s vlastnostmi spekter gama záření získaných polovodičovým spektrometrem.
    Měření se provádí na spektrometru KJF s GeLi detektorem o objemu aktivní oblasti 55 cm$^3$ (průměr čela detektoru je 70 mm). 
    Měření je prováděno se zářiči s jednoduchým spektrem gama záření: $^{137}$Cs (E = 661.55 keV), $^{60}$Co (E = 1173.27 keV) a 
    $^{24}$Na (E = 1368.63 a 2754.03 keV), které jsou současně používány ke kalibraci spektrometru. K nastavení geometrue zážič-detektor 
    se používá jednoduchý nosič zářičů umožňující volbu různé geometrie.
\end{enumerate}

\section{Teoretický úvod}
Gama záření je typ radiaktivního záření, které je tvořeno fotony o vysokých energiích. Je to jeden ze způsobý, kterým radiaktivní izotopy snižují svou energii. K jeho detekci se používají spektrometry např z polovodičů, ako při této úloze, kde toto záření vytrhává elektrony z mříže díky čemuž dochází ke změně el. proudu v měřícím obvodu. Mimo samotného chtěného efektu na detektoru se v naměřeném spektru projevuje spousta vedlejších efektů, které jsou podrobně pospány v \cite{text}.

\section{Měření}
\subsection{Kalibrace}
Nejprve jsem provedli kalibraci spektrometru za pomoci Ra, jehož spektrum je dobře známo.
Ke kalibraci jsme využili 8 bodů spektra. Chyba se v případě píků pohybovala okolo jednoho keV. 
V případě určování hran je však tato chyba až o řád vyšší.

\begin{figure}
\begin{center}
\input{spect1.tex}
\end{center}
\caption{Naměřené spektrum Cs izotopu 137.}
\label{spect1}
\end{figure}

\subsection{Cs}
Dále jsme na spektrometr umístili Cs izotop 137 a pozorovali vzniklé spektrum, které je vidět na obrázku \ref{spect1}. 
Na tomto spektru je vidět jediný dominantní pík, který odpovídá energii gama zážení Cs. Odpovídá energii
\begin{eqnarray}
E=661.66 \mbox{keV},
\end{eqnarray}
což zcela přesně koresponduje teoretické hodnotě.
\begin{eqnarray}
E_{Na}=661.66\mbox{keV}
\end{eqnarray}

Dále můžeme na spektru pozorovat projevy Comptnova rozptylu. Comptnovy hrany se nachází na energiích
\begin{eqnarray}
E_{h1}=480 \mbox{keV} \\
E_{h2}=552 \mbox{keV}
\end{eqnarray}
Hrana zpětného odrazu je na energii
\begin{eqnarray}
E_{hz}=195 \mbox{keV}
\end{eqnarray}
Dle teorie by hodnoty těchto hran měli být
\begin{eqnarray}
E_{th1}=477.33 \mbox{keV} \\
E_{th2}=554.58 \mbox{keV}
\end{eqnarray}

\begin{figure}
\begin{center}
\input{spect2.tex}
\end{center}
\caption{Naměřené spektrum radioaktivní soli.}
\label{spect2}
\end{figure}

\subsection{sůl}
Dále jsme pozorovali sůl ozařovanou neutronovým zářením. Tak v ní vznikly izotopy $^{24}$Na a radioaktivní izotop Cl.
Tam se mimo Comptnova rozpztylu navíc projevil efekt, při kterém z dvojice fotnů vzniká pár elektron, pozitron. Píky tohoto 
efektu jsou vyznačeny spolu s dalšímy body na obrázku \ref{spect2}. Jejich energetické hodnoty jsou
\begin{eqnarray}
E_{Na}=2753.82 \mbox{keV} \\
E_{NaSEP}=2242.77 \mbox{keV} \\
E_{NaDEP}=1731.84 \mbox{keV} \\
E_{Cl}=2167.28 \mbox{keV} \\
E_{ClSEP}=1642.51 \mbox{keV} \\
E_{Na2}=1368.50 \mbox{keV} \\
E_{Na2SEP}=856.83 \mbox{keV} \\
E_{ani}=510.95 \mbox{keV} \\
E_{J}=416.88 \mbox{keV}
\end{eqnarray}
Pro srovnání jsou teoretické energie 
\begin{eqnarray}
E_{tNa}=2754.03  \mbox{keV} \\
E_{tNaSEP}=2243.03 \mbox{keV} \\
E_{tNaDEP}=1732.03 \mbox{keV} \\
E_{tNa2}=1368.23  \mbox{keV} \\
E_{tNa2SEP}=857.23  \mbox{keV} \\
E_{tani}=511  \mbox{keV}
\end{eqnarray}
Pro úplnost jsem určil i hodnoty comptnových hran
\begin{eqnarray}
E_{hNa}=2521 \mbox{keV} \\
E_{h2Na}=2631 \mbox{keV} \\
E_{hNa2}=1157 \mbox{keV} \\
E_{h2Na2}=1253 \mbox{keV}
\end{eqnarray}
Pro srovnání opět uvádím teoretické hodnoty
\begin{eqnarray}
E_{thNa}=2520.22 \mbox{keV} \\
E_{th2Na}=2631.94 \mbox{keV} \\
E_{thNa2}=1152.93 \mbox{keV} \\
E_{th2Na2}=1251.39 \mbox{keV}
\end{eqnarray}

\section{Diksuze}
Na naměřených spektrech byli velmii dobře pozorovatelné všechny očekávané ekekty. Teoretické a tabulkové hodnoty se přesně  shodovali s naměřenými hodnotami dokonce s výrazně nižší chybou, než byla chyba fitu. Vliv pozadí byl řádově nižší než použité zářiče, takže neměli přílišný vliv na výsledky a proto ve spektrech nejsou žádné výrazné nadbytečné píky.

\section{Závěr}
Vyšetřil jsem spektra dvou různých zářičů, která jsou na obrázcích \ref{spect1} s \ref{spect2}, na kterých jsou také vyznačeny všechny významné body.

\begin{thebibliography}{5}
	\bibitem{text} \textbf{Studijní text na praktikum IV} \\http://physics.mff.cuni.cz/vyuka/zfp/txt\_400.pdf (9. 10. 2012)
    \bibitem{chyba} \emph{J. Englich}: \textbf{Zpracování výsldků fyzikálních měření} \\ LS 1999/2000
\end{thebibliography}
\end{document}
