\documentclass[a4paper,12pt]{article}
\usepackage{czech}                    
\usepackage[utf8]{inputenc}         
\usepackage{a4wide}                   
\usepackage[dvipdfm]{graphicx}        
\usepackage{indentfirst}   
\usepackage{fancyhdr}                 
\usepackage{setspace}
\usepackage{amsmath}
\usepackage{amssymb}
\usepackage{epsfig}
\usepackage[usenames]{color}

\begin{document}
\input epsf

\section{Pracovní úkol}
\begin{enumerate}
\item Změřte dynamickou viskozitu parafinového oleje Stokesovou metodou.
\item Změřte dynamickou viskozitu ricinového oleje Stokesovou metodou.
\item Ověřte, zda jsou pro dané experimentální uspořádání splněny podmínky platnosti Stokesova vzorce pro odpor prostředí při pohybu koule.
\item Hustotu skleněných kuliček určete pyknometrickou metodou. 
\end{enumerate}

\section{Teoretický úvod}

\subsection{Stokesova metoda měření viskozity}
Na pohybující se se těleso v tíhovém poli působí obecně tři síly - tíhová 
síla určena vztahem $G = m g$ (standardně $m$ je hmotnost tělesa, $g$ je tíhové zrychlení),
vztlaková síla a odporová síla. Vztlaková síla má dle Archimedova zákona 
tvar $Fvz = V \varrho g$, kde $V$ je objem tělesa a $\varrho$ je hustota tekutiny.
Pokud se jedná o pohyb kulově symetrického tělesa a jsou splněny podmínky 
laminárního obtékání popisuje odporovou sílu Stokesův vztah
\begin{eqnarray}
F = 6\pi \eta r v \, , \label{stokes}
\end{eqnarray}
kde $\eta$ je dynamická viskozita kapaliny, $r$ je poloměr tělesa, $v$ je 
jeho rychlost. Kromě laminarity proudění jsou zde předpokládány neomezené 
rozměry prostředí v němž se kulička pohybuje.

Kriterium laminarity proudění je určeno Reynoldsovým číslem $Re$, které je 
definováno 
\begin{eqnarray}
Re = \frac{2r\varrho v}{\eta}\, , \label{reynolds}
\end{eqnarray}
kde $\varrho$ je hustota tekutiny. Pokud je $Re << 1$ jedná se dle [1] o  
laminární proudění. 

Pokud se tedy jedná o pád kuliček ve válcové nádobě splňující podmínku 
laminárního proudění je viskozita tekutiny v ní dle [1] určena 
\begin{eqnarray}
\eta = \frac{2r^2(\varrho_k - \varrho)g}{9v\left(1-\frac{2,4r}{R}\right)} \, , \label{viskozita}
\end{eqnarray}
kde $\varrho_K$ je hustota kuliček, $R$ je poloměr válcové nádoby a $\varrho_k$ je husota kuliček..

\subsection{Pyknometrická metoda pro určení hustoty pevných látek}
Pro určení hustoty malých objektů určíme jejich hmotnost $m_1$ na vzduchu. 
Poté určíme hmotnost $m_2$ celého pyknometru naplněného kapalinou o hustotě 
$\varrho_z$. Nakonec určíme hmotnost $m_3$ tak, že do pyknometru naplněného 
vodou nasypeme objekty o hmotnosti $m_1$. Pak je tedy výsledná hustota tělísek 
$\varrho_k$ dle [5]
\begin{eqnarray}
\varrho_k = \frac{m_1}{m_1+m_2-m_3}(\varrho_z-\varrho_{vzduch})+\varrho_{vzduch} \, , \label{pyknometr}
\end{eqnarray}
kde $\varrho_{vzduch}$ je hustota vzduchu.


\section{Měření}

\subsection{Parametry kuliček}


%%Pro několik prvních kuliček (velkých i malých), 
%%jsem provedla více měření. U těchto hodnot jsem dle [2] provedla statistické zpracování 
%%a určila chybu dle [2]. Z časových důvodů jsem takové množství měření nemohla provést pro 
%%více kuliček. Proto s jistou ztrátou přesnosti budu jako další chybu brát chybu průměrnou.
Kuličky jsou dvou typů - velké a malé. Od každého typu jsem si vybrala 
10 kuliček. Pro ty jsem určila průměr pomocí dilenského mikroskopu. Naměřené hodnoty 
jsou v tabulce 2 (je zde uvedeno pouze 7 kuliček velkých - ostatní nebyly využity). 
Pro první tři velké kuličky a první dvě malé kuličky 
jsem provedla vždy po pěti měření (tabulka 1) a dle [2] jsem určila směrodatné odchylky, které jsem 
kvadraticky sečetla s polovinou nejmenšího dílku přístroje, tj $0,005 \, \mathrm{mm}$ . Pro ostatní 
kuličky jsem z časových důvodů tolik měření nemohla provést, proto jako chybu 
určení průměru ostatních kuliček beru aritmetický průměr statisticky určených chyb. Kuličky jsem si odkládala do číslovaných boxů. 
V dalším měření, jsem pak mohla rozlišovat jednotlivé velikosti kuliček.

Hustotu kuliček jsem určila pyknometrickou metodou. Tedy za předpokladu 
konstatní hustoty měřeného materiálu jsem vybrala soubor kuliček s nímž jsem 
pak prováděla měření nezávisle na číslovaných kuličkách. Naměřené hodnoty 
jsou v tabulce 7. Chyby jsou spočteny dle kvadratického 
hromadění chyb. Chyby jednotlivých hmotností jsem brala jako 
velikost předposledního digitu na měřících vahách, tedy $0,001 \, \mathrm{g}$.
Váhy mi na posledním desetinném místě i po doporučeném čase $3 \, \mathrm{s}$
přišly nestabilní.

Hustotu $\varrho_z$ destilované vody jsem odečetla z přítomné tabulky po 
určení teploty $t_z =(23,0 \pm 0,1)^{\circ}\mathrm{C}$ jako $\varrho = 997,3 \, \mathrm{kg \cdot m^{-3}}$.

Dále jsem určila atmosferický tlak $p_A = (987,0 \pm 2) \, \mathrm{hPa}$ a 
teplotu vzduchu $t_v =(24,8 \pm 0,4)^{\circ}\mathrm{C}$. Tyto chyby byly 
odečteny z dokumentace měřících přístrojů. Na základě těchto hodnot určím 
z [2] pomocí stavové rovnice hustotu vzduchu $\varrho_{vzduch} = (1,15 \pm 0,02) \, \mathrm{kg \cdot m^{-3}}$, 
kde pro určení chyb jsem kvadraticky sečetla relativní chyby teploty a tlaku.
Dosazením do (\ref{pyknometr}) pro hustoty kuliček dostanu\footnote{index $v$ načí velké kuličky, index $m$ značí malé kuličky}
\begin{eqnarray}
\nonumber \varrho_{k}^{V} &=& (2506 \pm 40) \, \mathrm{kg \cdot m^{-3}}, \\
\nonumber \varrho_{k}^{M} &=& (2469 \pm 53) \, \mathrm{kg \cdot m^{-3}},
\end{eqnarray}
kde chyba těchto veličin byla určena standardními vztahy pro rozdíl resp. podíl
relativních chyb z [2].

\subsection{Měření viskozity}

%%plot [0:20] 2**3*(1-exp(-x/2**2)),1**3*(1-exp(-x/1**2))
Do odměrného válce naplněného kapalinou jsem postupně pouštěla jednotlivé 
kuličky. Měřila jsem čas $t$, za který kulička urazila vzdálenost $l_r$ nebo 
$l_p$ \footnote{index $r$ značí ricínový olej, index $p$ značí parafínový olej}.
Při předpokladu, že v měřeném úseku je rychlost kuličky již maximální možná rychlost,
kterou při daném průměru v dané kapalině může dosáhnout, určím rychlost pádu kuličky 
ze vztahu pro rovnoměrný pohyb. Naměřené hodnoty jsou v tabulkách 3, 4, 
5 a 6.

Hustota kapalin je uvedena přímo u měřených kapalin jako $\varrho_r = 950 \, \mathrm{kg \cdot m^{-3}}$ a
$\varrho_p = 850 \, \mathrm{kg \cdot m^{-3}}$. 

Pro počáteční charakteristiku proudění v jednotlivých kapalinách jsem pomocí 
tabulkových hodnot viskozit [3], tedy $\eta_r = (986 \cdot 10^3) \, \mathrm{kg \cdot m^{-1} s^{-1}}$, 
$\eta_p = (101,8 \cdot 10^3) \, \mathrm{kg \cdot m^{-1} s^{-1}}$, určila hodnotu Reynoldsova 
čísla. Pro ricinový olej se jedná vždy o proudění laminární. Pro olej parafinový 
byla podmínka hodnoty \ref{reynolds} pro velké kuličky porušena. Proto jsem provedla 
pouze 2 měření pro ověření teoretického výpočtu.

Dosazením hodnot do vztahu \ref{viskozita} jsem určila hodnotu viskozity pro každou 
kuličku zvlášť. Následně jsou určila jejich střední hodnotu, tedy 
\begin{eqnarray}
\nonumber \eta_{r}^{V} &=& (832 \pm 37 ) \, \mathrm{kg \cdot mm^{-1} s^{-1}},\\
\nonumber \eta_{r}^{M} &=& (732 \pm 52 ) \, \mathrm{kg \cdot mm^{-1} s^{-1}},\\
\nonumber \eta_{p}^{V} &=& (46 \pm 4 ) \, \mathrm{kg \cdot mm^{-1} s^{-1}},\\
\nonumber \eta_{p}^{V} &=& (61 \pm 4 ) \, \mathrm{kg \cdot mm^{-1} s^{-1}},
\end{eqnarray}
kde chyby byly určeny postupem z [2] a [4] pomocí kvadratického sčítání relativních chyb a statistického 
zpracování souboru hodnot. Chybu určení času jsem na stopkách s přesností $0,01 \mathrm{s}$ odhadla 
i s reakční dobou na $0,1 \mathrm{s}$.

\section{Diskuze}
\subsection{Parametry kuliček}
Při měření hustoty kuliček pyknometrem se předpokládá, že hustota
souboru je vzhledem k jejich výrobnímu postupu u všech kuliček stejná.
Pokud by jejich hustoty takové nebyly určili bychom pouze průměrnou hustotu
celého souboru a tím bychom vnesli chybu do celkového výsledku.

U měření průměrů kuliček je vidět jejich značná různorodost. Navíc
všechny kuličky nemají přesně kulový tvar a tedy pro ně neplatí Stokesův
vztah. Celkový výsledek měření je na určení průměru kuliček velmi citlivý,
proto by se přesnost zvýšila větším počtem měření pro každou kuličku.

\subsection{Měření viskozity}
Výsledné hodnoty se s hodnotami tabulkovými ([3]) neshodují ani v rámci
určené chyby. To může být částečně způsobeno rozdílným složením měřených látek a
částečně také teplotou. Tabulková hodnota byla udána pro $20^{\circ}\mathrm{C}$. 
Nicméně se neshodují v rámci chyby ani určené viskozity pro jednotlivé velikosti 
kuliček. 

Rozdílná hodnota výsledků při měření parafínového oleje se vzhledem k turbulentnímu 
charakteru proudění v případě velkých kuliček očekávala. Ale u oleje 
ricinového se dle hodnoty \ref{reynolds} pro velké i malé kuličky  
jednalo o laminární proudění a tedy hodnoty by se neměli příliš lišit.
 
Důvodem rozdílnosti těchto hodnot by mohl být nerovnoměrný pohyb na měřeném 
úseku. Další chybu vnáší již zmiňovaná tvarová nepravidelnost kuliček a tedy 
neoprávněné užití Stokesova vztahu. Také vztah \ref{viskozita}, který platí 
přesně při pádu kuličky středem válce, čehož nešlo vždy přesně docílit.

\section{Závěr}
Změřila jsem průměry kuliček dilénským mikroskopem. Naměřené hodnoty 
zobrazují tabulky 1 a 2.

Pyknometrickou metodou jsem určila hustu malých i velkých kuliček. 
Naměřené hodnoty jsou v tabulce 7. Výsledné hodnoty 
husotot jsou 
\begin{eqnarray}
\nonumber \varrho_{k}^{V} &=& (2506 \pm 40) \, \mathrm{kg \cdot m^{-3}}, \\
\nonumber \varrho_{k}^{M} &=& (2469 \pm 53) \, \mathrm{kg \cdot m^{-3}}.
\end{eqnarray}

Dále jsem určila dynamickou viskozitu racínového a parafínového oleje pro
obě velikosti kuliček. Tedy \footnote{Kde dolní index značí druh kapaliny (r - racínový olej, p - parafínový olej) a horní index velikost kuličky.}
\begin{eqnarray}
\nonumber \eta_{r}^{V} &=& (832 \pm 37 ) \, \mathrm{kg \cdot mm^{-1} s^{-1}},\\
\nonumber \eta_{r}^{M} &=& (732 \pm 52 ) \, \mathrm{kg \cdot mm^{-1} s^{-1}},\\
\nonumber \eta_{p}^{V} &=& (46 \pm 4 ) \, \mathrm{kg \cdot mm^{-1} s^{-1}},\\
\nonumber \eta_{p}^{V} &=& (61 \pm 4 ) \, \mathrm{kg \cdot mm^{-1} s^{-1}}.
\end{eqnarray}
Hodnoty jsou v tabulkách 4-6.

\begin{thebibliography}{99}
\bibitem{stud_text} Studijní text - Volný pád koule ve viskózní kapalině,\\ http://physics.mff.cuni.cz/vyuka/zfp/txt\_119.pdf, 13.3 2011
\bibitem{tab} Mikulčák, J. a kol. - Matematické, fyzikální a chemické tabulky, Prometheus, Praha 2007, 1. vydání
\bibitem{broz} Brož,J a kol. - Fyzikální a matematické tabulky, SNTL, Praha 1980
\bibitem{ciz} Čížek, J. - Úvod do praktické fyziky, http://physics.mff.cuni.cz/kfnt/, 13.3 2011
\bibitem{cdfg} Studijní text - Pyknometrická metoda pro určení hustoty pevných látek, \\ http://physics.mff.cuni.cz/vyuka/zfp/txt\_119\_pyknometr.pdf,13.3 2011
\end{thebibliography}

%%%%%%% TABULKY %%%%%%%%
\begin{table}[!htb]
	\label{prumery_malo}
     \caption[]{Průměry kuliček u kterých bylo provedeno více měření} 
   \begin{center}
    \begin{tabular}{|c|c|c|c|c|c|}
        \hline
	&	$d_{1^V} /\mathrm{mm}$&	$d_{2^V} /\mathrm{mm}$	&	$d_{3^V} /\mathrm{mm}$	&	$d_{1^M} /\mathrm{mm}$	&	$d_{1^M} /\mathrm{mm}$ \\
\hline \hline										
	&	2,656	&	2,585	&	2,57	&	1,465	&	1,54 \\ \cline{2-6}
										
	&	2,58	&	2,56	&	2,58	&	1,425	&	1,525\\ \cline{2-6}
									
	&	2,57	&	2,575	&	2,555	&	1,46	&	1,54\\ \cline{2-6}
										
	&	2,59	&	2,605	&	2,565	&	1,48	&	1,525\\ \cline{2-6}
										
	&	2,575	&	2,57	&	2,56	&	1,46	&	1,52\\
\hline	\hline									
$\overline{d} /\mathrm{mm}$	&	$2,59 \pm 0,06$	&	$2,58 \pm 0,03$	&	$2,59 \pm 0,02$	&	$1,46 \pm 0,04$	&	$1,53 \pm 0,02$\\
\hline																					
    \end{tabular}
     \end{center}
  \end{table}
  
\begin{table}[!htb]
	\label{prumery}
     \caption[]{Výsledné průměry kuliček} 
   \begin{center}
    \begin{tabular}{|c|c|}
        \hline
$d_{k^V} /\mathrm{mm}$	&	$d_{k^M} /\mathrm{mm}$\\
\hline		\hline
2,59	&	1,46\\
\hline		
2,58	&	1,53\\
\hline		
2,57	&	1,53\\
\hline		
2,57	&	1,54\\
\hline		
2,62	&	1,52\\
\hline		
2,50	&	1,57\\
\hline		
2,54	&	1,59\\
\hline		
	&	1,57\\
\hline		
	&	1,53\\
\hline		
	&	1,50\\
	\hline
    \end{tabular}
     \end{center}
  \end{table}
  
  \begin{table}[!htb]
	\label{viskozitar-velke kulicky}
     \caption[]{Určení viskozity - velké kuličky ricinový olej} 
   \begin{center}
    \begin{tabular}{|c|c|c|}
        \hline
        $t_{r}^V/\mathrm{s}$	& $v_{r}^V/\mathrm{cm\cdot s^{-1}}$	&	$\eta_{r}^{V}/\mathrm{kgmm^{-1}\cdot s^{-1}}$ \\
\hline		\hline		
26,4	&	0,62	&	844,932\\
\hline				
25,85	&	0,64	&	818,026\\
\hline				
26,19	&	0,63	&	820,762\\
\hline				
26,48	&	0,62	&	832,343\\
\hline				
26,03	&	0,63	&	845,990\\
\hline				
    \end{tabular}
     \end{center}
  \end{table}
  
    \begin{table}[!htb]
	\label{viskozitar-male kulicky}
     \caption[]{Určení viskozity - malé kuličky ricinový olej} 
   \begin{center}
    \begin{tabular}{|c|c|c|}
        \hline
        $t_{r}^M/\mathrm{s}$	& $v_{r}^M/\mathrm{cm\cdot s^{-1}}$	&	$\eta_{r}^{M}/\mathrm{kgmm^{-1}\cdot s^{-1}}$ \\
\hline		\hline	
  67,79	&	0,24	&	692,121\\
\hline				
66,9	&	0,25	&	750,529\\
\hline				
66,56	&	0,25	&	746,715\\
\hline				
66,37	&	0,25	&	749,345\\
\hline				
65,03	&	0,25	&	720,262\\
\hline
      \end{tabular}
     \end{center}
  \end{table}
  
    \begin{table}[!htb]
	\label{viskozitap-male kulicky}
     \caption[]{Určení viskozity - malé kuličky parafinový olej} 
   \begin{center}
    \begin{tabular}{|c|c|c|}
        \hline
        $t_{p}^M/\mathrm{s}$	& $v_{p}^M/\mathrm{cm\cdot s^{-1}}$	&	$\eta_{p}^{M}/\mathrm{kgmm^{-1}\cdot s^{-1}}$ \\
        \hline \hline
3,05	&	5,44	&	63,448\\
\hline				
3,14	&	5,28	&	63,098\\
\hline				
3,2	&	5,18	&	60,399\\
\hline				
3,07	&	5,40	&	59,851\\
\hline				
3,16	&	5,25	&	56,298\\
\hline				
      \end{tabular}
     \end{center}
  \end{table}  
 
 \begin{table}[!htb]
	\label{viskozitap-velke kulicky}
     \caption[]{Určení viskozity - velké kuličky parafínový olej} 
   \begin{center}
    \begin{tabular}{|c|c|c|}
        \hline
        $t_{p}^V/\mathrm{s}$	& $v_{p}^V/\mathrm{cm\cdot s^{-1}}$	&	$\eta_{p}^{V}/\mathrm{kgmm^{-1}\cdot s^{-1}}$ \\
\hline		\hline		 
  1,32 &	12,56	&42,006\\
  \hline
1,58 &	10,49&	49,714\\
\hline
      \end{tabular}
     \end{center}
  \end{table}  
   \begin{table}[!htb]
	\label{pyknokul}
     \caption[]{Pyknometrické určení hustoty kuliček} 
   \begin{center}
    \begin{tabular}{|c||c|c|}
\hline   
 &velké kuličky & malé kuličky\\
 \hline \hline
 $m_1 \mathrm{g}$ & 4,77 & 3,49 \\
 \hline
  $m_2 \mathrm{g}$ & 11,66& 11,41 \\
   \hline
   $m_3 \mathrm{g}$ & 14,54&13,50 \\
    \hline
    \end{tabular}  
         \end{center}
  \end{table} 
\end{document}
