\documentclass[a4paper,12pt]{article}
\usepackage{czech}                    
\usepackage[utf8]{inputenc}         
\usepackage{a4wide}                   
\usepackage[dvipdfm]{graphicx}        
\usepackage{indentfirst}   
\usepackage{fancyhdr}                 
\usepackage{setspace}
\usepackage{amsmath}
\usepackage{amssymb}
\usepackage{epsfig}
\usepackage[usenames]{color}

\begin{document}
\input epsf

\section{Pracovní úkol}
\begin{enumerate}
\item Změřte tuhost aparatury $K$.
\item Proveďte dynamickou zkoušku deformace v tlaku přiloženého vzorku.
\item Výsledek dynamické zkoušky v tlaku graficky znázorněte a určete 
mezní napětí $\sigma_{0,2}$ a $\sigma_{U}$. 
\end{enumerate}


\section{Teoretický úvod}

\subsection{Deformace vzorku}
Při působení tlakové či tahové síly na vzorek dochází k jeho deformaci. 
Pro popis deformace je definováno skutečné napětí 
\begin{eqnarray} 
\nonumber \sigma^{'} = \frac{F}{S},
\end{eqnarray}
kde $F$ je působící síla kolmá na skutečný průřez $S$. Dále je definováno 
smluvní napětí
\begin{eqnarray} 
\sigma = \frac{F}{S_0}, \label{napeti}
\end{eqnarray}
kde $S_0$ je průřez nedeformovaného vzorku.

Dále je standardně zavedená relativní\footnote{je možné definovat také 
skutečnou deformaci určenou vztahem $\varepsilon = \ln{\frac{l}{l_0}}$} 
deformace
\begin{eqnarray}
\varepsilon_0 = \frac{l-l_0}{l_0} = \frac{\Delta l}{l_0}, \label{reldef}
\end{eqnarray}
kde $l$ resp. $l_0$ je rozměr deformovaného resp. nedeformovaného vzorku.

Průběh deformace v závislosti na velikosti působící síly (působícího napětí) 
probíhá pro různé materiály obecně velmi různě. Pro poměrně velké 
množství kovových materiálů je relativní deformace $\varepsilon_0$ přímo 
úměrná velikosti působícího napětí. Tedy až po nějaké mezní napětí lze 
deformaci popsat Hookovým zákonem. Napětí při kterém přestane platit 
přímá úměra se nazývá \textit{mez úměrnosti} $\sigma_{U}$. Při překročení 
tohoto napětí deformace probíhá stále elasticky. Při překročení \textit{meze pružnosti} 
$\sigma_E$ se deformace stává plastickou a i po uvolnění působícího napětí 
zůstane těleso trvale zdeformováno. Velikost konečné hodnoty trvalé deformace 
vzorku závisí na velikosti a na době po kterou napětí působí. 

Pro popis trvalé deformace se zavádí napětí $\sigma_{0,2}$, zváno též \textit{mez 0,2}. 
Je definováno jako napětí jemuž odpovídá plastická deformace o hodnotě 
$\varepsilon_{pl} = 0,2 \, \%$. Mez 0,2 lze dle [1] určit přímo ze zatěžovacího 
diagramu rozložením deformace na plastickou a elastickou část a nalezením 
na křivce v zatěžovacím diagramu bod odpovídající $\varepsilon_{pl} = 0,2 \, \%$. 
Tímto bodem $(\varepsilon_{pl} = 0,2 \, \%, \sigma = 0)$ a směrnicí danou 
Hookovým zákonem je určena lineární funkce. Hodnotu $\sigma_{0,2}$ lze 
pak odečíst z průsečíku výše zmiňované lineární funkce a  křivky  určené 
naměřenými hodnotami.

\section{Dynamická zkouška deformace v tlaku}
K měření deformace v tlaku slouží měřící aparatura, která stlačuje vzorek 
konstantní rychlostí a zaznamenává velikost působící síly a čas.
Pokud jedna otáčka kotouče aparatury odpovídá stlačení vzorku o $0,75 \, \mathrm{mm}$
a počet otáček normován časem má hodnoty $0,6 \cdot 10^{-3} \, s^{-1}$, pak je 
změna délky měřeného vzorku dána vztahem
\begin{eqnarray}
 \Delta l = 0,75 \cdot 0,6 \cdot 10^{-3} \cdot t, \label{dlprepocet}
\end{eqnarray}
kde $t$ je čas, kterému vždy přísluší údal o velikosti deformace. 

Během měření dochází k vlastní deformaci aparatury, tu je nutné odečíst 
od deformace vzorku. Tedy pro výslednou deformaci platí vztah
\begin{eqnarray}
\Delta l_V = 0,75 \cdot 0,6 \cdot 10^{-3} \cdot t - \frac{F}{K}, \label{dlfinal}
\end{eqnarray}
kde $K$ je vlastní tuhost aparatury a $F$ je síla na vzorek působící.
Pro tu platí
\begin{eqnarray}
F = \alpha \cdot U, \label{zUdoF}
\end{eqnarray}
kde U je výstupní napětí a $\alpha = 50 \, \mathrm{N/mV}$ je převodní 
konstanta charakterizující aparaturu mezi elektrickými a mechanickými veličinami.


\section{Měření}
\subsection{Tuhost aparatury}
Do aparatury jsem vložila kalibrační váleček z tuhé oceli, jehož rozměry 
jsou značně větší než rozměry testovacích vzorků. Tedy při postupném 
stlačováním kalibračního válečku zaznamenávám průběh deformace samotné 
aparatury v závislosti na působící síle. Výstup z měřící aparatury 
zajišťuje program Zapisovač jenž vypíše vždy čas měření a výstupní hodnotu
elektrického napětí. Přepočet mezi těmito veličinami určují vztahy (\ref{dlprepocet})
a (\ref{zUdoF}). Výslednou závislost zobrazuje obrázek \ref{kalibrace}. 
Tuhost určená lineární regresí v lineární části grafu (nezapočítaná data z 
počátku měření, viz obrázek \ref{kalibrace}) programem Gnuplot je
\begin{eqnarray}
\nonumber K= (1707 \pm 0,6) \, \mathrm{kN/m},
\end{eqnarray}
kde uvedená chyba je chyba lineární regrese určena programem.

Poté jsem vložila do aparatury vzorek, tj. váleček o počátečních rozměrech 
změřených mikrometrem
\begin{eqnarray}
\nonumber v &=& (10,20 \pm 0,01) \, \mathrm{mm},\\
\nonumber d &=& (7,48 \pm 0,01) \, \mathrm{mm},
\end{eqnarray}
za chybu měřidla jsem brala nejmenší dílek stupnice (v případě mikrometru $0,01 \, \mathrm{mm}$).

V případě měřeného vzorku jsem ke zpracování výstupních hodnot programu Zapisovač 
použila vztahy (\ref{dlfinal}),(\ref{zUdoF}) a přepočtu na relativní 
prodloužení a napětí vztahy (\ref{napeti}) a (\ref{reldef}). Výslednou 
závislost ukazuje obrázek \ref{vzorek}. 
Proložením lineární části grafu Hookovým zákonem dostanu modul pružnosti
válečku v tlaku $E = (1986 \pm 9 ) \, MPa$, což je zároveň směrnice přímky 
sloužící k určení $\sigma_{0,2}$. Dále je v grafu zaznamenána hodnota 
meze úměrnosti $\sigma_U$ a také hodnota $\sigma_{0,2}$.
Tedy tyto veličiny nabývají hodnot
\begin{eqnarray}
\nonumber \sigma_U = (5.7 \pm 0.5) \, MPa , \\
\nonumber \sigma_{0,2} = (9.8 \pm 0.1) \, MPa ,
\end{eqnarray}
kde chyba u $\sigma_U$ je určena z [2] jakožto velikost chyby elektrických
měřících přístrojů, z rozsahu měřícího přístroje a měřeného intervalu. Tato 
chyba vychází velmi malá, proto nadhodnocuji její velikost, vzhledem k tomu, 
že je nutné tuto hodnotu ručně odečíst z grafu.
Nejistota $\sigma_{0,2}$ je zatížena jak chybou určení průměru $d$ vzorku, 
ale také chybou fitu veličiny $E$. Tyto chyby jsem dle pravidel kvadratického 
hromadění chyb sečetla.

Rozměry po deformaci
\begin{eqnarray}
\nonumber v_d &=& (9,79 \pm 0,01) \, \mathrm{mm},\\
\nonumber d_d &=& (7,60 \pm 0,01) \, \mathrm{mm},
\end{eqnarray}

\section{Diskuze}
Z výsledků získaných lineární regresí při určování tuhosti aparatury, 
tedy jejich velkou přesností, lze
usuzovat, že se aparatura chová elasticky a nedochází k její trvalé deformaci.
Téměř v celém průběhu je deformace přímo úměrná působící síle.
Odchylka od lineární závislosti na začátku může být způsobena nerovnostmi
kalibračního vzorku. Odchylka ke konci měření může indikovat nelineární deformaci 
aparatury popřípadě kalibračního vzorku. Pro lineární regresi byla použita 
pouze ta data, která nejeví odchylku od lineární závislosti.

Při vlastním měření vzorku je na obrázku \ref{vzorek} vidět, že nejprve je relativní 
deformace přímo úměrná působící síle, ale s jinou směrnicí než následný úsek.
To může být způsobeno povrchovými nerovnostmi vzorku, které se po jisté 
době působení síly odstraní. Popřípadě počáteční různoběžností podstav válečku.
Dále je závislost lineární a splňuje Hookův 
zákon. Po překročení $\sigma_U$ (viz obrázek \ref{vzorek} je pozorovatelná 
nelineární elastická deformace a následně trvalá deformace plastická.

Určení meze pružnosti $\sigma_U$ je zatíženo velkou chybou, jedná se spíše o odhad. 
Pro přesnější určení této hodnoty by bylo třeba měření provést pro větší množství
válečků a lépe vyhodnotit zátěžový diagram daného materiálu.
Určení hodnoty $\sigma_{0,2}$ je naopak mnohem přesnější. Modul pružnosti 
v tahu je určen poměrně přesně a totéž platí o průměru válečku.

Z konečných rozměrů vzorku se potvrdilo, že došlo k trvalé plastické deformaci.
Na zdeformovaném vzorku je okem pozorovatelná nerovnoměrnost, tedy vzorek 
se nedeformoval v celém objemu rovnoměrně, což vnáší do měření další nepřesnost.


\section{Závěr}
Změřila jsem tuhost aparatury 
\begin{eqnarray}
\nonumber K= (1707 \pm 0,6) \, \mathrm{kN/m},
\end{eqnarray}
naměřená data s proloženou lineární závislost jsou na obrázku \ref{kalibrace}.

Dále jsem provedla dynamickou zkoušku deformace testovacího vzorku,
kde výsledné jsou zobrazeny na obrázku \ref{vzorek} a \ref{detail}.
Z těchto grafů jsem odečetla mez úměrnosti $\sigma_U$ a mez 0,2,
\begin{eqnarray}
\nonumber \sigma_U = (5.7 \pm 0.4) \, MPa , \\
\nonumber \sigma_{0,2} = (9.8 \pm 0.1) \, MPa.
\end{eqnarray}


\begin{thebibliography}{99}
\bibitem{stud_text} Studijní text - http://physics.mff.cuni.cz/vyuka/zfp/111.htm, Dynamická zkouška deformace látek v tlaku, 11.3 2011
\bibitem{ciz} Čížek, J. - Úvod do praktické fyziky, http://physics.mff.cuni.cz/kfnt/, 27.2 2011
\end{thebibliography}

\begin{figure}[!htb]
\input{kalibracefp.tex}
\caption{Kalibrace měřící aparatury}
\label{kalibrace}
\end{figure}


\begin{figure}[!htb]
\input{detailp.tex}
\caption{Deformace vzorku v měřící aparatuře}
\label{vzorek}
\end{figure}

\begin{figure}[!htb]
\input{detailopr.tex}
\caption{Detail deformace vzorku v měřící aparatuře}
\label{detail}
\end{figure}

\end{document}
