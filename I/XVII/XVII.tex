\documentclass[a4paper,12pt]{article}
\usepackage{czech}
\usepackage[utf8]{inputenc}
\usepackage{a4wide}
\usepackage[dvipdfm]{graphicx}
\usepackage{graphics}
\usepackage{indentfirst}
\usepackage{fancyhdr}
\usepackage{setspace}
\usepackage{amsmath}
\usepackage{amssymb}
\usepackage{epsfig}

%%\usepackage{nopageno}
%%\usepackage{txfonts}
\usepackage[usenames]{color}

\begin{document}
\section{Úkol}
\noindent
\begin{enumerate}
	\item Změřte momenty setrvačnosti kvádru vzhledem k hlavním osám setrvačnosti
	\item Určete složky jednotkového vektory ve směru zadané osy rotace kvádru 
	v souřadné sousavě dané hlavními osami setrvačnosti.
	\item Vypočítejte moment setrvačnosti kvádru vzhledem k zadané obecné ose 
	rotace. Výsledek ověřte měřením.
	\item Měrně ověřte Steinerovu větu.
\end{enumerate}

\section{Teorie}
\noindent
Moment setrvačnosti $J$ je symetrický tenzor charakteryzující tuhé těleso. Každé těleso 
má tři na sebe kolmé osy, kdy je tento tenzo v diagonalizovaném tvaru. Tyto osy se 
nazývají hlavní a většinou se jedná o osy symetrie.

Pro symetrická tělesa jsou známy vztahy pro výpočet jejich momentu setrvačnosti. Např. 
moment setrvačnosti válce podle osy procházející těžistěm a rovnoběžné s výškou platí
\begin{eqnarray}
	J_V=\frac{1}{2}MR^2,
\end{eqnarray}
kde $M$ je hmotnost válce a $R$ jeho poloměr. Více naleznete např. v \cite{Kvasnica}.

Dalším důležitým vztahem pro určování momentu setrvačnosti pro jiné osy je Steinerova věta
\begin{eqnarray}
	J=J_0+md^2,
\end{eqnarray}
kde $J_0$ moment setrvačnosti těleda pro osu procházející těžištěm, $m$ hmotnost tělesa 
a $d$ vzdálenost os. Ta však platí pouze pro osy, které jsou vzájemně rovnoběžné.

Při kmitání tuhého tělesa na torzním vlákně je perioda tohoto pohybu dána vztahem
\begin{eqnarray}
	T=2\pi \sqrt{\frac{J}{D}},
\end{eqnarray}
kde $J$ je složka tenzoru setrvačnosti pro danou osu a $D$ deviační moment.
Pokud máme těleso se známým momentem setrvačnosti, můžeme změřit periodu jeho kmitání 
a pro naše neznámé těleso užijeme vztah
\begin{eqnarray}
	J=\frac{T^2}{T_T^2}J_T,
	\label{vypocet J}
\end{eqnarray}
který vychází z rovnosti deviačních momentů. 

Pro moment setrvačnosti vzhledem k zadané obecné ose platí vztah
\begin{eqnarray}
	J_v=v_x^2J_{0x}+v_y^2J_{0y}+v_z^2J_{0z},
\end{eqnarray}
kde $J_i$ jsou složky tenzoru setrvačnosti a $\vec v$ je jednotkový vektor definovaný 
\begin{eqnarray}
v_x=\frac{a}{\sqrt{a^2+b^2+c^2}},v_y=\frac{b}{\sqrt{a^2+b^2+c^2}},v_z=\frac{c}{\sqrt{a^2+b^2+c^2}},
\end{eqnarray}
kde $a$, $b$, $c$ jsou rozměry kvádru.

Pro těleso kývající se podle pevné osy platí vztah
\begin{eqnarray}
	T=2\pi\sqrt{\frac{J}{mgd}},
\end{eqnarray}
kde $J$ je moment setrvačnosti, $m$ hmotnost tělesa, $g$ tíhové zrychlení a $d$ 
vzdálenost těžiště od osy otáčení. Z něj můžeme vyjádřit vztah pro výpočet 
momentu setrvačnosti
\begin{eqnarray}
	J=mgd\left(\frac{T}{2\pi}\right)^2
\end{eqnarray}

Podrobný návod naleznete v \cite{text}.

\section{Výsledky měření}
\subsection{Moment setrvačnosti kvádru vzhledem k hlavním osám}
Mejprve jsem zvážil válec ($m_V$) a kvádr ($m_K$). Jejich hmotnosti jsou i s chybou 
určenou dle \cite{chyba}
\begin{eqnarray}
	m_V=(903.9\pm0.2)\mbox{g} \\
	m_K=(1072.0\pm0.2)\mbox{g} 
\end{eqnarray}

Poté jsem změřil průměr válce $D$ a rozměry kvádru, které jsou vyznačeny na obrázku 1. 
Výsledné hodnoty jsou
\begin{eqnarray}
	D=(108.06\pm0.02)\mbox{mm} \\
	a=(127.92\pm0.02)\mbox{mm} \\	
	b=(64.10\pm0.02)\mbox{mm} \\
	c=(18.81\pm0.02)\mbox{mm}
\end{eqnarray}

Následně jsem měřil dobu pěti nebo 10 period kmitů válce a kvádru podle různých os, 
přičemž délce $a$ odpovídá osa x, $b$ osa y a $c$ osa z. Naměřené hodnoty jsou shrnuty v tabulce 1. 
Počet měření jsem postupně snižoval, protože jejich vyšší počet by neměl výrazný 
vliv na velikost chyby.
$$
\begin{array}{|c|c|c|c|c|c|}
\hline
n&	5\cdot T_{0V}/\mbox{s}&		10\cdot T_{0x}/\mbox{s}&		5\cdot T_{0y}/\mbox{s}&		5\cdot T_{0z}/\mbox{s}&		5\cdot T_{0v}/\mbox{s}	\\ \hline
1&	55.03&				57.81&				52.77&				59.12&				35.33			\\ \hline
2&	54.98&				57.83&				52.88&				58.89&				35.00			\\ \hline
3&	55.20&				57.76&				52.71&				59.07&				35.46			\\ \hline
4&	55.10&				57.93&				52.82&				59.04&				35.32			\\ \hline
5&	55.19&				57.82&				52.68&				58.89&				35.20			\\ \hline
6&	55.06&				57.76&				52.87& & \\ \hline
7&	54.99&				57.71&				52.96& & \\ \hline
8&	55.01&				57.77&				52.87& & \\ \hline
9&	55.13&				57.80& & & \\ \hline
10&	54.96&				57.77& & & \\ \hline
\end{array}
$$
\vglue.5cm
\begin{center}
\textbf{Tabulka 1:} Periody kmitání podle různých os.
\end{center}

Měření jsme statisticky vyhodnotil dle \cite{chyba} a stanovil výsledné periody
\begin{eqnarray}
	T_{0V}=(11.01 \pm 0.04)\mbox{s} \\
	T_{0x}=(5.78 \pm 0.02)\mbox{s} \\
	T_{0y}=(10.56 \pm 0.04)\mbox{s} \\
	T_{0z}=(11.80 \pm 0.04)\mbox{s} \\
	T_{0v}=(7.05 \pm 0.04)\mbox{s} \\
\end{eqnarray}

Dle (1) jsem vypočetl moment setrvačnosti válce
\begin{eqnarray}
	J_{0V}=(528\pm4)\cdot 10^{-5} \mbox{kg}\cdot\mbox{m}^2.\\
\end{eqnarray}
Poté jsem dle (4) určil momenty setrvačnosti kvádru vzhledem k hlavním osám
\begin{eqnarray}
	J_{0x}=(145\pm1)\cdot 10^{-5} \mbox{kg}\cdot\mbox{m}^2,\\
	J_{0y}=(485\pm4)\cdot 10^{-5} \mbox{kg}\cdot\mbox{m}^2,\\
	J_{0z}=(606\pm5)\cdot 10^{-5} \mbox{kg}\cdot\mbox{m}^2.
\end{eqnarray}

\subsection{Moment setrvačnosti kvádru vzhledem k zadané ose}
\noindent
Podle vzorců (6) jsem stanovil složky jednotkového vektoru $\vec v$
\begin{eqnarray}
	v_x=0.886\pm0.001\\
	v_y=0.444\pm0.001\\
	v_z=0.130\pm0.001
\end{eqnarray}
a dopočital dle (5) moment setrvačnosti vzhledem k zadané ose
\begin{eqnarray}
	J_{v}=(72\pm2)\cdot 10^{-5} \mbox{kg}\cdot\mbox{m}^2.
\end{eqnarray}

Pro tuto osu jsem také určil moment setrvačnosti metodou uvedenou výše, kde je uvedena 
i perioda kmitání podle této osy $T_{0v}$
\begin{eqnarray}
	J_{0v}=(118\pm1)\cdot 10^{-5} \mbox{kg}\cdot\mbox{m}^2.
\end{eqnarray}

\subsection{Steinerova věta}
\noindent
Nejprve jsem mnou zkoumanou tyč zvážil
\begin{eqnarray}
	m_t=(281.6\pm0.2)\mbox{g}
\end{eqnarray}
a určil vzdálenost těžiště od břitu, kolem kterého se otáčí. Tato vzdálenost se špatně 
určovala, proto jsem měření opakovalm, přičemž naměřené hodnoty jsou uvedeny v tabulce 2.
$$
\begin{array}{|c|ccccc|}
\hline
d/\mbox{cm}&	15.7&	15.6&	15.6&	15.5&	15.6 \\
\hline
\end{array}
$$
\vglue.5cm
\begin{center}
	\textbf{Tabulka 2:} Vzdálenost těžiště od břitu.
\end{center}
Tyto hodnoty jsem dle \cite{chyba} vyhodnotil a výsledná vzdálenost je
\begin{eqnarray}
	d=(15.6\pm0.1)\mbox{cm}
\end{eqnarray}

Následně jsem dle postupu výše určil moment setrvačnosti tyče dle osy procházející těžištěm
\begin{eqnarray}
	J_{0t}=(1107\pm9)\cdot 10^{-5} \mbox{kg}\cdot\mbox{m}^2,
\end{eqnarray}
přičemž perioda kmitu tyče byla
\begin{eqnarray}
	T_{0t}=(15.95\pm0.01)\mbox{s}.
\end{eqnarray}
Tuto periodu jsem určil ze dvou měření deseti period, přičemž druhé měření bylo pouze pro 
vyloučení hrubé chyby způsobené např. špatným spočítáním kmitů.

Hodnotu $J_{0t}$ jsem přepočetl dle (2) pro osu procházející břitem
\begin{eqnarray}
	J_t=(179\pm1)\cdot 10^{-4} \mbox{kg}\cdot\mbox{m}^2.
\end{eqnarray}

Nakonec změřil dobu deseti kmitů kyvadla sestrojeného z tyče se břitem v ložisku. Tyto 
hodnoty jsou v tabulce 3.
\eject
$$ 
\begin{array}{|c|c|c|c|}
\hline
n&	10\cdot T_t/\mbox{s}&	n&	10\cdot T_t/\mbox{s}\\	\hline
1&	9.38&	6&	9.38 \\ \hline
2&	9.42&	7&	9.42 \\ \hline
3&	9.47&	8&	9.38 \\ \hline
4&	9.45&	9&	9.48 \\ \hline
5&	9.40&	10&	9.40 \\ \hline
\end{array}
$$
\vglue.5cm
\begin{center}
	\textbf{Tabulka 3:}Doba deseti kmitů kyvadla.
\end{center}
Výsledná perioda kyvadla je včetně chyby
\begin{eqnarray}
	T_{t2}=(0.86\pm0.09)\mbox{s}
\end{eqnarray}
a použitím vztahu (8) jsem získal memont setrvačnosti
\begin{eqnarray}
	J_{t2}=(800\pm200)\cdot 10^{-5} \mbox{kg}\cdot\mbox{m}^2.
\end{eqnarray}

\section{Diskuze}
\noindent
Mnou naměřené hodnoty se s těmi teoreticky vypočítanými značně liší. Chyba je zajisté 
v měření. Vzorec odvozený pro stanovení momentu setrvačnosti z torze vlákna počítá 
s ideálními podmínkami. Vlákno však bylo zjevně opotřebené častým používáním a v jednom 
místě značně defotmované. 

Další věc, se kterou jsem nepočítal byl úchyt vlákna. Jednalo 
se o šroub, který se upevnil do zkoumaného tělesa. I při pečlivém utažení však zajisté 
docházelo k povolování závitu, což se projevilo na výsledcích měření. Samotné měření 
času tak napřesné nebylo. 

Chyba způsobená rekční dobou a nepřesností určení přesné periody 
se na tak velkém časovém pseku neprojevila. 

Poslední chyba, která vznikla při určování 
momentu setrvačnosti podle obecné osy vznikla tak, že zkoumané těleso ve skutečnosti kvádr 
nebylo. Bylo výrazně zkosené v rozích.

Z mých výpočtů také vyplývá, že Steinerova věta neplatí, o čemž pochybuji a problém je opět 
na strane správného určení momentu setrvačnosti. Měření periody kyvadla již tak přesné nebylo 
a ve vzorci je jsou opět zanedbány vnější vlivy, odporové síly atd.

\section{Závěr}
\noindent
Změřil jsem momenty setrvačnosti tělesa vzhledem k hlavním osám
\begin{eqnarray}
	J_{0x}=(145\pm1)\cdot 10^{-5} \mbox{kg}\cdot\mbox{m}^2,\\
	J_{0y}=(485\pm4)\cdot 10^{-5} \mbox{kg}\cdot\mbox{m}^2,\\
	J_{0z}=(606\pm5)\cdot 10^{-5} \mbox{kg}\cdot\mbox{m}^2.
\end{eqnarray}
Určil jsem složky jednotkového vekoru zadané osy v souřadném systému hlavních os
\begin{eqnarray}
	v_x=0.886\pm0.001,\\
	v_y=0.444\pm0.001,\\
	v_z=0.130\pm0.001.
\end{eqnarray}
Určil jsem moment setrvačnosti vzhledem k zadané ose jak výpočtem
\begin{eqnarray}
	J_{v}=(72\pm2)\cdot 10^{-5} \mbox{kg}\cdot\mbox{m}^2,
\end{eqnarray}
tak měřením
\begin{eqnarray}
	J_{0v}=(118\pm1)\cdot 10^{-5} \mbox{kg}\cdot\mbox{m}^2.
\end{eqnarray}
Nepodařilo se mi ověřit platnost Steinerovy věty. Naměřená hodnota momentu setrvačnosti byla
\begin{eqnarray}
	J_{t2}=(800\pm200)\cdot 10^{-5} \mbox{kg}\cdot\mbox{m}^2
\end{eqnarray}
a dle této věty vypočtená
\begin{eqnarray}
	J_t=(179\pm1)\cdot 10^{-4} \mbox{kg}\cdot\mbox{m}^2.
\end{eqnarray}



\begin{thebibliography}{5}
	\bibitem{text} \textbf{Studijní text na praktikum I} \\http://physics.mff.cuni.cz/vyuka/zfp/txt\_117.pdf (8. 4. 2011)
	\bibitem{Kvasnica} \emph{Prof. RNDr. Jozef Kvasnica, DrSc. a kolektiv}: \textbf{Mechanika}\\ Academia, Praha 1988
        \bibitem{chyba} \emph{J. Englich}: \textbf{Zpracování výsldků fyzikálních měření} \\ LS 1999/2000
\end{thebibliography}
\end{document}
