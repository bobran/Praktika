\documentclass[a4paper,12pt]{article}
\usepackage{czech}
\usepackage[utf8]{inputenc}
\usepackage{a4wide}
\usepackage[dvipdfm]{graphicx}
\usepackage{graphics}
\usepackage{indentfirst}
\usepackage{fancyhdr}
\usepackage{setspace}
\usepackage{amsmath}
\usepackage{amssymb}
\usepackage{epsfig}

%%\usepackage{nopageno}
%%\usepackage{txfonts}
\usepackage[usenames]{color}

\begin{document}

\section{Úkol}
\begin{enumerate}
    \item Změřte divergenci laserového svazku.
    \item Z optické stavebnice sestavte Michelsonův interferometr. K rozšíření svazku sestavte Galileův teleskop. Ze známých ohniskových délek použitých čoček spočtěte, kolikrát bude laserový svazek rozšířen a porovnejte s naměřenou hodnotou.
    \item Pozorujte interferenční proužky při změně polohy zrcadla $\mbox{Z}_3$, vysvětlete pozorovaný efekt. Do jednoho z interferujících svazků vložte některé z přiložených skel. Popište a vysvětlete změny v interferenčním obrazci. 
\end{enumerate}

\section{Teorie}
\subsection{Divergence paprsku}
Divergence paprsku je definována vztahem
\begin{eqnarray}
d=\frac{D_2-D_1}{s},
\label{d}
\end{eqnarray}
kde $D_1$ je průměr paprsku na začátku a $D_2$ průměr po uražení dráhy $s$. Minimální dosažitelná divergence je dána vztahem
\begin{eqnarray}
d_m\approx \frac{2\lambda}{D_1},
\label{dm}
\end{eqnarray}
kde $\lambda$ je vlnová délka daného paprsku.

\subsection{Galileův teleskop}
Galileův teleskop se skládá z rozptylky a spojky, které mají jedno společné ohnisko. Jeho zvětšení je dáno vztahem
\begin{eqnarray}
Z=\frac{f_1}{f_2}.
\label{Z}
\end{eqnarray}

\subsection{Interference}
Interference vzniká z důvodu fázového rozdílu světla dopadajícího do jednoho bodu, kde se vlny skládájí. U světla ze stejného zdroje (stejnou počáteční fází) je fázový nábéh způsoben rozdílem drahovým. Pomínky pro maximum a minimum jsou
\begin{eqnarray}
2(l_2-l_1)=k\lambda \\
2(l_2-l_1)=k\lambda + \lambda/2
\end{eqnarray}
Více naleznete v \cite{maly}.

\section{Měření}

\subsection{Divergence laserového svazku}
Nejprve jsem změřil divergenci laserového svazku. Vzhledem k nepřesnosti při měření průměru stačilo pro měření dráhy pouze jedno měření
\begin{eqnarray}
s=(255.0\pm0.1)\mbox{cm}
\end{eqnarray}
U průměru zase byla chyba měřidla tak výrazná, že opakování měření nemělo smysl.
\begin{eqnarray}
D_1=(1.0\pm0.5)\mbox{mm} \\
D_2=(6.0\pm0.5)\mbox{mm}
\end{eqnarray}
Po dosazení do zvtahu \ref{d} získáme
\begin{eqnarray}
d=0.0020 \pm 0.0004.
\end{eqnarray}
Pro srovnání, minimální dosažitelnou divergencí pro laser o $\lambda=632.8$ nm je
\begin{eqnarray}
d_m=0.0013
\end{eqnarray}

\subsection{Sestavení stavebnice}
Dle návodu v \cite{text} jsem postupně přidával jednotlivé optické součástky. Po přidání každé jsem vždy 
doladil výsku paprsku, a to za pomoci obrazu, který dopadal na stínítko u zdroje. Po seřízení zrcadel jsem 
nejprve přidal mezi zrcadla Z$_1$ a Z$_2$ rozptylku, kterou jsem seřídil tak, aby světlo z ní vychátející dopadalo 
na střed zrcadla Z$_2$. Nastavení jsem upravoval podle obrazu na stínítku za zrcadlem Z$_2$. Následně jsem do vzdálenosti 
175 mm umístil spojnou čočku. Tento odhad byl určen dle přibližných ohniskových vzdáleností použitých čoček, které byly
\begin{eqnarray}
f_1=-25\mbox{mm} \\
f_2=200\mbox{mm}
\end{eqnarray}
Následně jsem doostřil tak, aby obraz na stínítku u zdroje byl stejný jako před přidáním čoček.

Při toumto rozložení jsem mezi zrcadly Z$_2$ a Z$_3$ změřil průměr svazku
\begin{eqnarray}
D'_2=(9.0 \pm 0.5) \mbox{mm},
\end{eqnarray}
což dle vztahu \ref{Z} dává zvětšení
\begin{eqnarray}
Z=9 \pm 1
\end{eqnarray}
Teoretická hodnota zvětšení pro čočky s parametry uvedenými výše je
\begin{eqnarray}
Z_t=8
\end{eqnarray}

Nakonec jsem přidal dělič svazku a poslední zracadlo. Výsledný interferenční obrazec jsem ještě za pomoci spojné čočky zvětšil.

\subsection{Interferenční obrazec}
Na vzniklém obraze byly jasně viditelné interferenční obrazce. Vyzkoušel jsem jiich chování při posunu zrcadla Z$_3$ a při různé 
manipulaci se zrcadly. Dále jsem zkusil účinek různých sklíček a ohževu vzduchu v místě průchod paprsku. Výsledky jsou shrnuty v diskuzi.

\section{Dikuze}
Naměřená hodnota divergence svazku je spíše pro představu, že se paprsek příliš nerozbíhá, než že by se dala nějak použít. Trpí totiž 
velkou chybou, která je způsobena nevhodností milimetrového papíru na měření řádově milimetrového paprsku. Pro představu je však zcela dostačující, 
protože se k žádným výpočtům nepoužívá.

Zvětšení koresponduje s teoretckou hodnotou.Tomu také nasvědčoval fakt, že nebylo třeba při doostřování příliš měnit vzdálenost mezi čočkami. 
Odchylka je způsobena opět chybou milimterového papíru.

Interferenční obrazce nebylo příliš sto upravovat za pomoci posuvu zrcadel. Při sebejemnějším pohybu byly změny příliš velké na to, aby se dali pozorovat. 
K pohybu totiž sloužil mikrometrický šroub, kde posun o jeden dílek byl o dva řády vyšší než vzdálenost potřebná ke zmněně z minima na maximum. Obrazec se 
proto nohem lépe upravoval za pomoci natáčení jednoho ze zrcadel, při čemž vnikal vírazně menší drahový rozdíl. V závislosti na vzájemné poloze stop paprsku 
se měnila nejen vzdálenost proužků, ale i jejich vzálemná orientace. Proužky byly vždy kolmé na spojnici středů stop papsrsku a s rostoucí vzáleností se proužky 
zhušťovaly. Nejmenší mnou dosažený výsledek byly tři proužky. Přiložená sklíčka lokálně deformavala proužky. Ve skle totiž světlo cestuje pomaleji a v důsledku 
nehomogenity v tloušťce sklíček tak vzniká další drahový rozdíl. U tlustčího sklíčka byl jev samozřejmně zřetelnější. Podobný efekt mělo zahřátí  vzduchu pouze 
za pomoci ruky. Vlivem vyšší teploty se vzruch zřiďuje, čímž v něm lehce stoupá rychlost světla. Tento efekt měl za následek vlnění proužků. Při použití zapalovače 
pak vznikla ostrá hrana, která se opět mírně vlnila.

Celá aparatura byla poměrně náchylná na jakékoliv otřesy. To občas zplsobilo rozostření obrazu či lehčí změnu na interferenčních proužcích. Výsledné 
závěry to však nijak neovlivnilo, naopak mnohdy navodilo nové situace.

\section{Závěr}
\noindent
Změřil jsem divergenci laserového paprsku
\begin{eqnarray}
d=0.0020 \pm 0.0004.
\end{eqnarray}
Sestavil jsem Mikelsonův interferometr a změřil rozšíření paprsku
\begin{eqnarray}
Z=9 \pm 1
\end{eqnarray}
Pozoroval jsem změnu interferenčního obrazce v závislosti na různých vlivech.

\eject
\begin{thebibliography}{5}
	\bibitem{text} \textbf{Studijní text na praktikum III} \\http://physics.mff.cuni.cz/vyuka/zfp/txt\_320.htm (8. 3. 2012)
    \bibitem{chyba} \emph{J. Englich}: \textbf{Zpracování výsldků fyzikálních měření} \\ LS 1999/2000
    \bibitem{maly} \emph{prof. RNDr. Petr Malý , DrSc.}: \textbf{Optika}\\Univerzita Karlova v Praze, Nakladatelství Karolinum 2008, první vydání
\end{thebibliography}




\end{document}
